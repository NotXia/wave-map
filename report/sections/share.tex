
\subsection{Condivisione dati}

Le operazioni per l'esportazione e l'importazione dei dati sono implementate come metodi statici della classe \texttt{ShareMeasures}.

La condivisione dei dati avviene tramite la creazione di un file contenente le misurazioni.
Per maggiore interoperabilità con eventuali estensioni, i dati vengono salvati in formato JSON e per tale scopo viene utilizzata la libreria \texttt{Gson}.

Per facilitare future espansioni delle funzionalità di condivisione, la navigazione tra i fragment dedicati alla condivisione è stato implementato utilizzando Navigation component.

\subsubsection{Esportazione}
L'esportazione è gestita dal fragment \texttt{FileExportFragment} e ViewModel \texttt{FileExportViewModel}.

Durante la fase di esportazione, tutte le misurazioni e tutti i BSSID salvati nel database vengono convertiti in una stringa JSON, assieme ad alcuni metadati. Il risultato è quindi temporaneamente salvato nella cache dell'applicazione.

L'utente ha quindi la possibilità di salvare il risultato in un file locale nella cartella \textit{Downloads} oppure di condividerlo attraverso un \texttt{Intent} di tipo \texttt{ACTION\_SEND}.


\subsubsection{Importazione}
L'importazione è gestita dall'activity \texttt{ImportActivity} e ViewModel \texttt{ImportViewModel}.

\texttt{ImportActivity} ha come \texttt{intent-filter} azioni del tipo \texttt{VIEW} e \texttt{SEND}, e quindi permette di selezionare l'azione di importazione quando l'utente apre o condivide un file.

Una volta aperto e validato un file di misurazioni, l'importazione consiste nell'inserire nel database tutte le misurazioni presenti marcandole come misure ottenute per condivisione. Per evitare duplicati, l'UUID viene usato come discriminante e vengono quindi ignorate tutte le misurazioni già presenti.
Inoltre, per evitare valori troppo sparsi, se il timestamp di una misurazione importata è molto vicino rispetto ad una misurazione locale, viene uniformato con quello della misurazione locale.

Analogamente, la tabella dei BSSID viene importata escludendo le righe già note.