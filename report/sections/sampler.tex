\subsection{Raccolta dei dati}

\subsubsection{Struttura e memorizzazione delle misurazioni}
Una misurazione è descritta dall'interfaccia \texttt{WaveMeasure} e contiene il valore della misurazione, un timestamp, la posizione e un flag per indicare se si tratta di una misurazione propria o ottenuta tramite condivisione. 
In aggiunta, è presente un campo per informazioni aggiuntive utile per distinguere alcune tipologie di misurazioni (es. per Wi-Fi e Bluetooth viene salvato il BSSID).

L'interfaccia \texttt{WaveMeasure} viene quindi utilizzata per implementare la classe \texttt{MeasureTable} che descrive la tabella del database dedicata alla memorizzazione delle misurazioni. 
Tutte le misurazioni sono salvate nella stessa tabella e sono differenziate da un campo (\texttt{type}) aggiunto in fase di salvataggio nel database.
Inoltre, ciascuna misura è identificata univocamente da un UUID, utile anche per differenziare le misurazioni provenienti da altri utenti.

In aggiunta, una seconda tabella descritta dalla classe \texttt{BSSIDTable} contiene la mappatura da BSSID a SSID.


\subsubsection{Sampler}
Per la raccolta dei dati è stato introdotto il concetto di \textit{sampler} per gestire in maniera modulare le misurazioni.
Nello specifico, un \textit{sampler} è descritto dalla classe astratta \texttt{WaveSampler} e richiede l'implementazione dei seguenti metodi:
\begin{itemize}
    \item \texttt{sample} per prendere una nuova misurazione.
    \item \texttt{store} per il salvataggio dei dati nel database.
    \item \texttt{retrieve} per la ricerca dei dati note le coordinate dei vertici di una cella della mappa.
\end{itemize}
Inoltre, sono esposte le seguenti funzioni ausiliarie:
\begin{itemize}
    \item \texttt{average} richiama \texttt{retrieve} e restituisce la media dei valori.
    \item \texttt{sampleAndStore} richiama in sequenza \texttt{sample} e \texttt{store}.
\end{itemize}
Per maggiore flessibilità, le misure vengono sempre intese come liste di \texttt{WaveMeasure}. Ciò permette di gestire misurazioni che per loro natura non generano un'unica misurazione (es. Wi-Fi e Bluetooth).

A partire da \texttt{WaveSampler} sono quindi implementati i \textit{sampler} per:
\begin{itemize}
  \item Wi-Fi (\texttt{WiFiSampler}): 
    \begin{itemize}[topsep=0em]
      \item Ottiene la potenza della rete al quale il dispositivo è attualmente connesso tramite il servizio di sistema \texttt{ConnectivityManager}. Per versioni inferiori alla API 29, viene invece utilizzato il \texttt{WifiManager}.
      \item Misura la potenza delle reti circostanti registrando un \texttt{BroadcastReceiver} con filtro \texttt{WifiManager.SCAN\_RESULTS\_AVAILABLE\_ACTION} e richiedendo una scansione completa attraverso il \texttt{WifiManager}.
    \end{itemize}
  \item Bluetooth (\texttt{BluetoothSampler}):
    \begin{itemize}[topsep=0em]
      \item Misura la potenza dei dispositivi accoppiati mediante il \texttt{BluetoothManager}.
      \item Misura la potenza dei dispositivi circostanti registrando un \texttt{BroadcastReceiver} con filtro \texttt{BluetoothDevice.ACTION\_FOUND} e richiedendo al \texttt{BluetoothManager} una scansione completa.
    \end{itemize}
  \item LTE (\texttt{LTESampler}): ottiene la potenza del segnale LTE tramite il \texttt{TelephonyManager}.
  \item Suono (\texttt{NoiseSampler}): viene fatta la media di una serie di campionature effettuate utilizzando un \texttt{MediaRecorder}.
\end{itemize}