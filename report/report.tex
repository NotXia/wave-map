\documentclass[11pt]{article}
\usepackage[margin=3cm]{geometry}
\usepackage{algorithm2e}
\usepackage[italian]{babel}
\usepackage[hidelinks]{hyperref}

\tolerance=1
\emergencystretch=\maxdimen
\hyphenpenalty=10000
\hbadness=10000
\setlength\parindent{0pt}

\begin{document}
\begin{titlepage}
    \begin{center}
        \vspace*{5cm}
            
        \Huge
        \textbf{Cellular Connectivity and\\Noise Map}
            
        \vspace{0.5cm}
        \LARGE
        Relazione
            
        \vspace{1cm}
          
		\hfill
		\begin{center}
        	{\large{\bf Xia $\cdot$ Tian Cheng}}\\[-0.2em]
			{\large Matricola: \texttt{0000975129}}\\[-0.2em]
			{\large Email: tiancheng.xia@studio.unibo.it}
        \end{center}
            
        \vspace{4cm}
            
        Anno accademico\\
        $2022 - 2023$
            
        \vspace{0.8cm}
            
            
        \Large
        Corso di Laboratorio di applicazioni mobili\\
        Alma Mater Studiorum $\cdot$ Università di Bologna\\
            
    \end{center}
\end{titlepage}
\newpage

\pagenumbering{roman}
\tableofcontents
\newpage

\pagenumbering{arabic}


\section{Introduzione}


\section{Scelte progettuali}

\subsection{Informazioni generali}



\subsection{Mappa}



\subsection{Raccolta dei dati}

\subsubsection{Struttura e memorizzazione delle misurazioni}
Una misurazione è descritta dall'interfaccia \texttt{WaveMeasure} e contiene il valore della misurazione, un timestamp e la posizione. 
In aggiunta, è presente un campo per informazioni aggiuntive utile per distinguere alcune tipologie di misurazioni (es. per Wi-Fi e Bluetooth viene salvato il BSSID).

L'interfaccia \texttt{WaveMeasure} viene quindi utilizzata per implementare la classe \texttt{MeasureTable} che descrive la tabella del database dedicata per memorizzare le misurazioni. 
Tutte le misurazioni sono salvate nella stessa tabella e sono differenziate da un campo \texttt{type}.


\subsubsection{Sampler}
Per la raccolta dei dati è stato introdotto il concetto di \textit{sampler} per gestisce in maniera modulare le misurazioni.
Nello specifico, un \textit{sampler} è descritto dalla classe astratta \texttt{WaveSampler} e richiede l'implementazione dei seguenti metodi:
\begin{itemize}
    \item \texttt{sample} per prendere una nuova misurazione
    \item \texttt{store} per il salvataggio dei dati nel database
    \item \texttt{retrieve} per la ricerca dei dati note le coordinate dei vertici di una cella della mappa
\end{itemize}
Inoltre, sono esposte le seguenti funzioni ausiliarie:
\begin{itemize}
    \item \texttt{average} richiama \texttt{retrieve} e restituisce la media dei valori
    \item \texttt{sampleAndStore} richiama in sequenza \texttt{sample} e \texttt{store}
\end{itemize}
Per maggiore flessibilità, le misure vengono sempre intese come liste di \texttt{WaveMeasure}. Ciò permette di gestire misurazioni che per loro natura non generano un'unica misurazione (es. Wi-Fi e Bluetooth).

A partire da \texttt{WaveSampler} sono quindi implementati i \textit{sampler} per Wi-Fi, Bluetooth, LTE e suono.


\end{document}
